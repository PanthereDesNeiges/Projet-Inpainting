\documentclass[12pt]{article}
\usepackage[utf8]{inputenc}
\usepackage{amsmath}
\usepackage{amssymb}
\usepackage[T1]{fontenc}    
\usepackage[utf8]{inputenc}   % Accents codés dans la font
\usepackage[french]{babel}  % Les traductions françaises
\usepackage{csquotes}
\usepackage{stmaryrd}
\usepackage{caption} 
\usepackage{geometry}
\usepackage{lmodern}
\usepackage{textcomp} 
\usepackage{mathrsfs}
\usepackage{latexsym}
\usepackage{amsfonts}
\usepackage{theorem}
\usepackage{graphicx}
\usepackage{scrlayer-scrpage}
\usepackage{xcolor}
\usepackage{setspace}
\usepackage{framed}
\usepackage{hyperref} 
\usepackage{multicol}
\usepackage{multirow}
\usepackage{xurl}
\usepackage{tabularx}
\usepackage{enumitem}
\usepackage{float}
\usepackage{bbold}
\usepackage{eurosym}
\usepackage{algorithm} 
\usepackage{algpseudocode} 
\usepackage{multicol}
\usepackage{verbatim}
\usepackage{pdfpages}
\usepackage[sorting=none]{biblatex}
\bibliography{biblio.bib}

\providecommand{\keywords}[1]
{
  \small	
  \textbf{\textit{Keywords---}} #1
}

\makeatletter
\newcommand{\pushright}[1]{\ifmeasuring@#1\else\omit\hfill$\displaystyle#1$\fi\ignorespaces}
\newcommand{\pushleft}[1]{\ifmeasuring@#1\else\omit$\displaystyle#1$\hfill\fi\ignorespaces}
\makeatother

\usepackage{subcaption}
\usepackage{graphicx}
\usepackage[export]{adjustbox}
\usepackage{wrapfig}
\setlength{\footheight}{50pt}

\geometry{
 a4paper,
 height = 257mm,
 right=25.4mm,
 left=25.4mm,
 top=25.4mm,
 head=25.4pt,
 }

\definecolor{gro}{gray}{2} % define color
\addtokomafont{footsepline}{\color{gro}} % define footer horizontal line
\ofoot{{\includegraphics[scale=.05]{logos/ponts.png} }} % footer (c=center)
\ifoot{Trucage photographique par Inpainting}


\begin{document}

\begin{titlepage}
\newcommand{\HRule}{\rule{\linewidth}{0.5mm}}
\newcommand{\hRule}{\rule{\linewidth}{0.1mm}}
\center
%\textsc{\LARGE
%Ecole Nationale des Ponts et Chaussées
%} \\[1cm]
\includegraphics[scale=0.16]{logos/ponts.png} \\[1cm]

École des Ponts ParisTech\\
2022 - 2023\\[0.5cm]

\HRule \\[0.4cm]
{ \huge \bfseries Rapport de Projet \\[0.15cm] }
\HRule \\[1.5 cm]

\textbf{Philomène BOISNARD} \\[0.2cm]
\textbf{Erwann ESTEVE} \\[0.2cm]
\textbf{Wandrille FLAMANT} \\[0.2cm]
\textbf{Sixtine NODET} \\[0.2cm]
Élèves ingénieurs\\[1.5cm]

\textbf{Projet de Semestre 2} \\ [0.5cm]
\textbf{\Large TRUCAGE PHOTOGRAPHIQUE PAR INPAINTING
} \\[2.5cm]

\textbf{Février - Juin 2023} \\[3cm]

Encadrant : \\ 
\textbf{Pascal MONASSE}\\[0.8cm]

\end{titlepage}

\begin{comment}
\vspace{0.15\textheight}
\begin{figure}[ht]
   \begin{minipage}{\textwidth}
     \includegraphics[width = \linewidth]{images/couverture.png}
     \raggedleft \scriptsize \textit{Source : chronopost.fr}
   \end{minipage}\hfill
\end{figure}
\end{comment}

%%%%%%%%%%%%%%%%%%%%%%%%%%%%%%%%%%%%%%%%%%%%%%%%%%
%%%%%%%%%%%%%%%%%%%%%%%%%%%%%%%%%%%%%%%%%%%%%%%%%%
%%%%%%%%%%%%%%%%%%%%%%%%%%%%%%%%%%%%%%%%%%%%%%%%%%
\newpage

%%%%%%%%%%%%%%%%%%%%%%%%%%%%%%%%%%%%%%%%%%%%%%%%%%
%%%%%%%%%%%%%%%%%%%%%%%%%%%%%%%%%%%%%%%%%%%%%%%%%%
%%%%%%%%%%%%%%%%%%%%%%%%%%%%%%%%%%%%%%%%%%%%%%%%%%
\tableofcontents
\setcounter{tocdepth}{2}

%%%%%%%%%%%%%%%%%%%%%%%%%%%%%%%%%%%%%%%%%%%%%%%%%%
%%%%%%%%%%%%%%%%%%%%%%%%%%%%%%%%%%%%%%%%%%%%%%%%%%
%%%%%%%%%%%%%%%%%%%%%%%%%%%%%%%%%%%%%%%%%%%%%%%%%%


\newpage
\section*{Introduction}
\label{sec:intro}
\addcontentsline{toc}{section}{\nameref{sec:intro}}

\paragraph{Notre projet}

se base principalement sur le papier de A.Criminisi \cite{criminisi2004region}.

Dans un premier temps, notre travail s'est concentré sur la compréhension 
du document et de l'algorithme expliqué.

\vspace{7pt}

L'\textbf{inpainting} est le nom donné à la technique de reconstruction 
d'images détériorées ou de remplissage des parties manquantes d'une image\\[0.5cm]

\textbf{L'objectif principal} de notre projet va être de construire un 
algorithme capable d'effectuer ce trucage par inpainting, idéalement 
dans une complexité spatiale et temporelle maîtrisée, puis d'en trouver les limites.



% Commentaires de Monasse pour le rapport:
%une dizaine de pages. Comme l'article qu'on a lu en gros
%faut pas parler de la partie programmation (ou seulement en annexe)
%il faut expliquer le fonctionnemet gloable de l'algorithme, le pseudo code
%des notations mathématiques.
%pas entrer dans les détails du code.

%Sources bien citées

%Chaque figure est numérotée + une légende claire qui permet de comprendre
%la figure meme si on a pas lu le rapport
%Chaque figure est citée.




%%%%%%%%%%%%%%%%%%%%%%%%%%%%%%%%%%%%%%%%%%%%%%%%%%
%%%%%%%%%%%%%%%%%%%%%%%%%%%%%%%%%%%%%%%%%%%%%%%%%%
%%%%%%%%%%%%%%%%%%%%%%%%%%%%%%%%%%%%%%%%%%%%%%%%%%
\newpage
\section{Présentation de l'algorithme}



De la lecture du document \cite{criminisi2004region}, nous en avons déduit 
l'\textbf{algorithme 1} suivant en pseudo-code  qui permet de résoudre 
le problème du trucage par inpainting.

\begin{algorithm}
    \caption{Algorithme de remplissage de la zone target}\label{alg:cap}
    \begin{algorithmic}
        \Require Image, target zone $\Omega$
        \State Initialiser les pixels hors de $\Omega$
        \State Initialiser de la frontière $\partial \Omega$
        \While{$\partial \Omega\neq\emptyset$}
            \State Calculer les priorités $P(p)$, $\forall p \in \partial \Omega$
            \State Trouver le pixel $ p \in \partial \Omega$ maximimsant $P(p)$
            \State Trouver le patch $\Psi_{q}$ minimisant la distance avec $\Psi_{p}$
            \State Copier $\Psi_{q}$ sur $\Psi_{p} \cap \Omega$
            \State Calculer $C(p)$, $\forall p \in \Psi_{p} \cap \Omega $
            \State Mettre à jour $\Omega$ et $\partial \Omega $
        \EndWhile
    \end{algorithmic}
\end{algorithm}



\vspace{0.3cm}

Nous avons fragmenté le travail en plusieurs sous problèmes que nous 
avons ensuite répartit entre les différents membres du groupe.

Les sous problèmes sont notamment ceux du traitement de la frontière 
de la \textit{target zone}, de la gestion des paramètres \texttt{confidence} 
et \texttt{data}, l'affichage des images, le matching d'un \textit{patch}, etc.

%%%%%%%%%%%%%%%%%%%%%%%%%%%%%%%%%%%%%%%%%%%%%%%%%%
%%%%%%%%%%%%%%%%%%%%%%%%%%%%%%%%%%%%%%%%%%%%%%%%%%
\subsection{Les objets exploités}

La \textit{target zone} $\Omega$, un patch $\Psi_p$, la frontière $\partial\Omega$. \\[0.3cm]

Pour mieux comprendre les objects appartenant aux classes \texttt{Pixel}, 
\texttt{Image} et \texttt{Frontiere}, les choix d'implémentation 
sont développés \hyperref[les-classes]{en Annexe}.

%%%%%%%%%%%%%%%%%%%%%%%%%%%%%%%%%%%%%%%%%%%%%%%%%%
%%%%%%%%%%%%%%%%%%%%%%%%%%%%%%%%%%%%%%%%%%%%%%%%%%
\subsection{L'initialisation}

\subsubsection{Initialisation des pixels hors de la zone $\Omega$}

Texte


\subsubsection{Initialisation de la frontière $\partial\Omega$}

Texte


%%%%%%%%%%%%%%%%%%%%%%%%%%%%%%%%%%%%%%%%%%%%%%%%%%
%%%%%%%%%%%%%%%%%%%%%%%%%%%%%%%%%%%%%%%%%%%%%%%%%%

\subsection{Calcul des termes \textit{Priorité, Confiance} et \textit{Data}}

\subsubsection{Calcul des priorités $P(p)$}

Texte

\subsubsection{Calcul de la Confiance (ou \textit{Confidence}) $C(p)$}

Texte

\subsubsection{Calcul du terme \textit{Data} $D(p)$}

Texte

\subsection{Le \textit{Matching} et le calcul des distances}

Texte


\subsection{Mise à jour de la \textit{target zone} $\Omega$ et de la 
frontière $\partial\Omega$}

Texte


%%%%%%%%%%%%%%%%%%%%%%%%%%%%%%%%%%%%%%%%%%%%%%%%%%
%%%%%%%%%%%%%%%%%%%%%%%%%%%%%%%%%%%%%%%%%%%%%%%%%%
%%%%%%%%%%%%%%%%%%%%%%%%%%%%%%%%%%%%%%%%%%%%%%%%%%
\newpage
\section{Analyse des résultats}

%%%%%%%%%%%%%%%%%%%%%%%%%%%%%%%%%%%%%%%%%%%%%%%%%%
%%%%%%%%%%%%%%%%%%%%%%%%%%%%%%%%%%%%%%%%%%%%%%%%%%
\subsection{Sous-partie}

Texte

%%%%%%%%%%%%%%%%%%%%%%%%%%%%%%%%%%%%%%%%%%%%%%%%%%
%%%%%%%%%%%%%%%%%%%%%%%%%%%%%%%%%%%%%%%%%%%%%%%%%%
\subsection{Sous-partie}

Texte


%%%%%%%%%%%%%%%%%%%%%%%%%%%%%%%%%%%%%%%%%%%%%%%%%%
%%%%%%%%%%%%%%%%%%%%%%%%%%%%%%%%%%%%%%%%%%%%%%%%%%
\subsection{Sous-Partie}

Texte



%%%%%%%%%%%%%%%%%%%%%%%%%%%%%%%%%%%%%%%%%%%%%%%%%%
%%%%%%%%%%%%%%%%%%%%%%%%%%%%%%%%%%%%%%%%%%%%%%%%%%
%%%%%%%%%%%%%%%%%%%%%%%%%%%%%%%%%%%%%%%%%%%%%%%%%%
\newpage
\section{Critique de l'algorithme}


%%%%%%%%%%%%%%%%%%%%%%%%%%%%%%%%%%%%%%%%%%%%%%%%%%
%%%%%%%%%%%%%%%%%%%%%%%%%%%%%%%%%%%%%%%%%%%%%%%%%%
\subsection{La complexité}

Spatiale et temporelle

Texte

%%%%%%%%%%%%%%%%%%%%%%%%%%%%%%%%%%%%%%%%%%%%%%%%%%
%%%%%%%%%%%%%%%%%%%%%%%%%%%%%%%%%%%%%%%%%%%%%%%%%%
\subsection{Les cas limites}

Texte


%%%%%%%%%%%%%%%%%%%%%%%%%%%%%%%%%%%%%%%%%%%%%%%%%%
%%%%%%%%%%%%%%%%%%%%%%%%%%%%%%%%%%%%%%%%%%%%%%%%%%
\subsection{Pistes d'amélioration}

Texte




%%%%%%%%%%%%%%%%%%%%%%%%%%%%%%%%%%%%%%%%%%%%%%%%%%
%%%%%%%%%%%%%%%%%%%%%%%%%%%%%%%%%%%%%%%%%%%%%%%%%%
%%%%%%%%%%%%%%%%%%%%%%%%%%%%%%%%%%%%%%%%%%%%%%%%%%
\newpage
\section*{Conclusion}
\label{sec:conclu}
\addcontentsline{toc}{section}{\nameref{sec:conclu}}

\paragraph{Retour d'expérience.} 

\paragraph{Delta.} 

\paragraph{Améliorations.}


%%%%%%%%%%%%%%%%%%%%%%%%%%%%%%%%%%%%%%%%%%%%%%%%%%
%%%%%%%%%%%%%%%%%%%%%%%%%%%%%%%%%%%%%%%%%%%%%%%%%%
%%%%%%%%%%%%%%%%%%%%%%%%%%%%%%%%%%%%%%%%%%%%%%%%%%

\newpage
\addcontentsline{toc}{section}{Références}

\printbibliography



%%%%%%%%%%%%%%%%%%%%%%%%%%%%%%%%%%%%%%%%%%%%%%%%%%
%%%%%%%%%%%%%%%%%%%%%%%%%%%%%%%%%%%%%%%%%%%%%%%%%%
%%%%%%%%%%%%%%%%%%%%%%%%%%%%%%%%%%%%%%%%%%%%%%%%%%

\newpage
\appendix
\section*{Annexes}
\label{sec:annexes}
\addcontentsline{toc}{section}{\nameref{sec:annexes}}

\subsection*{Les classes principales}\label{les-classes}

\subsubsection*{La classe \texttt{Pixel}}

Texte

\subsubsection*{La classe \texttt{Image}}

Texte

\subsubsection*{La classe \texttt{Frontière}}

Texte

%%%%%%%%%%%%%%%%%%%%%%%%%%%%%%%%%%%%%%%%%%%%%%%%%%
%%%%%%%%%%%%%%%%%%%%%%%%%%%%%%%%%%%%%%%%%%%%%%%%%%
\subsection*{Captures d'écran}
\label{sub:images}
\addcontentsline{toc}{subsection}{\nameref{sub:images}}

Les bo screens waa

% \begin{figure}[ht]
%   \centering
%   \begin{minipage}{0.99\textwidth}
%      \centering
%      \includegraphics[width=\linewidth]{images/calendrier_cours.png}
%   \end{minipage}\hfill
%   \caption{Calendrier, onglet « Mon emploi du temps »}
%   \label{fig:impoved_calendar}
% \end{figure}

% \begin{figure}[ht]
%   \centering
%   \begin{minipage}{0.99\textwidth}
%      \centering
%      \includegraphics[width=\linewidth]{images/tous_les_cours.png}
%   \end{minipage}\hfill
%   \caption{Liste des cours}
%   \label{fig:touscours}
% \end{figure}

\end{document}